% Tomado del Anteproyecto, adaptar
\textit{
    En este Trabajo Profesional desarrollaremos un videojuego en línea con la particularidad de que el servidor del juego será un sistema distribuido bajo el paradigma del modelo de actores.
Para el backend, utilizaremos las herramientas que provee el toolkit de Akka para implementar un servidor distribuido bajo el paradigma de actores. Además haremos uso de Akka Streams para el manejo de actualizaciones de estado de juego generales, reservando el uso de mensajes de actores para interacciones entre las entidades.
Para el frontend (cliente), implementaremos un programa o aplicación de escritorio que pueda comunicarse con el servidor y renderice el juego en base a las actualizaciones que va recibiendo. Este cliente va a permitir la interacción del usuario con otros jugadores y entidades del juego.
El objetivo que perseguimos no es meramente el de permitir la interacción de múltiples jugadores en línea en base al modelo de actores, sino desarrollar una arquitectura que permita la conexión concurrente de una gran cantidad de jugadores, priorizando la escalabilidad y disponibilidad del sistema.
    }
