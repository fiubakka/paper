% En esta sección se debe presentar el problema detectado y/o oportunidad 
% de mejora relacionada con la construcción de la aplicación presentada. 
% La misma debe ser redactada de manera clara y concisa debe ser entendida 
% en la primera lectura, este ítem es de fundamental importancia al momento 
% de ser evaluado el proyecto.

Si bien el modelo monolítico descrito anteriormente funciona bien para la mayoría de los 
casos con una cantidad acotada de jugadores, empieza a ser un problema cuando el juego 
permite la interacción de una magnitud mayor de jugadores / entidades (NPCs) en simultáneo. 
Al ser una arquitectura monolítica las herramientas típicas que se utilizan para escalar 
la performance es delegar ciertas tareas a distintos threads del procesador, para así distribuir 
el cómputo realizado y conseguir paralelizar las tareas, obteniendo una mejor performance. 
Sin embargo el uso de threads conlleva conocidos potenciales problemas de concurrencia, 
como pueden ser los deadlocks o race conditions que se originan al estar compartiendo 
recursos / memoria. Además, este enfoque se ve también limitado por la cantidad de threads 
que posea el procesador, siendo imposible paralelizar más tareas que la cantidad de núcleos 
que el procesador posea.
