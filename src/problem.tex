% En esta sección se debe presentar el problema detectado y/o oportunidad 
% de mejora relacionada con la construcción de la aplicación presentada. 
% La misma debe ser redactada de manera clara y concisa debe ser entendida 
% en la primera lectura, este ítem es de fundamental importancia al momento 
% de ser evaluado el proyecto.

\noindent Si bien el modelo monolítico descrito anteriormente funciona bien para la mayoría de los 
casos con una cantidad acotada de jugadores, empieza a ser un problema cuando el juego 
permite la interacción de una magnitud mayor de jugadores u otras entidades en simultáneo. 
Para mejorar la performance se suele recurrir a delegar ciertas tareas a distintos hilos del procesador, para así distribuir 
el cómputo realizado y conseguir paralelizar el procesamiento de la lógica del juego. 
Sin embargo, el uso de hilos conlleva conocidos problemas de concurrencia. Entre estos se encuentran los \textit{deadlocks} o \textit{race conditions} que se originan al estar compartiendo 
recursos, principalmente la memoria. Este enfoque se ve también limitado por la cantidad de hilos 
que posea el procesador, donde la cantidad dada nos marca el límite máximo posible de paralelización
del que se dispone. Además, si el juego no fue diseñado con concurrencia en mente, puede resultar imposible separarlo en secciones lógicas
independientes que puedan ser procesadas en paralelo.

Todas estas problemáticas resultan en limitantes a la hora de determinar la cantidad máxima de eventos
que se pueden procesar en una única instancia del juego, manteniendo un nivel de performance aceptable para el flujo del juego.
Si el juego a desarrollar tiene como requisito soportar una cantidad elevada de jugadores
en simultáneo será necesario buscar una solución alternativa que permita escalar horizontalmente.
