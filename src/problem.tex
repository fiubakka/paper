% En esta sección se debe presentar el problema detectado y/o oportunidad 
% de mejora relacionada con la construcción de la aplicación presentada. 
% La misma debe ser redactada de manera clara y concisa debe ser entendida 
% en la primera lectura, este ítem es de fundamental importancia al momento 
% de ser evaluado el proyecto.

\noindent Si bien el modelo monolítico descrito anteriormente funciona bien para la mayoría de los 
casos con una cantidad acotada de jugadores, empieza a ser un problema cuando el juego 
permite la interacción de una magnitud mayor de jugadores / entidades en simultáneo. 
Al ser una arquitectura monolítica las herramientas típicas que se utilizan para escalar 
la performance es delegar ciertas tareas a distintos hilos del procesador, para así distribuir 
el cómputo realizado y conseguir paralelizar el procesamiento de la lógica del juego, obteniendo una mejor performance. 
Sin embargo, el uso de hilos conlleva conocidos problemas de concurrencia. Entre estos se encuentran los \textit{deadlocks} o \textit{race conditions} que se originan al estar compartiendo 
recursos, principalmente la memoria. Este enfoque se ve también limitado por la cantidad de hilos 
que posea el procesador, donde la cantidad dada nos marca el límite máximo posible de paralelización
del que se dispone.

Todas estas problemáticas resultan en limitantes a la hora de determinar la cantidad máxima de jugadores
que se pueden procesar en una única instancia del juego, manteniendo un nivel de performance aceptable para el flujo del juego.
Si el juego a desarrollar tiene como requisito soportar una cantidad elevada de jugadores
en simultáneo, es necesario buscar una solución que permita escalar horizontalmente y es aquí donde proponemos
una alternativa.

