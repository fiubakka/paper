% En esta sección se deben explicar los riesgos que se materializaron a lo largo del
% proyecto, estuvieran o no previstos al comienzo (más aún si no estaban previstos).
% Por ejemplo, desvíos en las estimaciones que hayan obligado a modificar el alcance,
% interesados con los que esperaban contar que finalmente no estuvieron disponibles, 
% metodologías o tecnologías que no funcionaron como se esperaba, y cuestiones de 
% este estilo.
% Los riesgos no previstos son per se lecciones aprendidas, pero también resulta de 
% interés explicitar qué aspectos del proyecto les dejaron lecciones positivas que 
% hayan servido para el aprendizaje.

\noindent A continuación detallamos los problemas encontrados a lo largo del desarrollo del
Trabajo Profesional. Estos problemas fueron en su mayoría imprevistos que o bien
dificultaron el progreso o bien imposibilitaron el desarrollo de ciertas partes.

\subsection{Deploy de Kubernetes a un ambiente productivo}

\subsection{Shelved game features}

\subsection{Godot Editor}