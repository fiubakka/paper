\noindent El presente documento abarca una gran cantidad de conceptos y tecnologías que son necesarios comprender
para poder tener un completo entendimiento del sistema desarrollado. Comenzaremos describiendo la problemática que decidimos
abordar para contextualizar la situación actual del estado del arte en el área de estudio. Presentaremos luego nuestra alternativa,
los diferentes componentes que la conforman, y cómo se relacionan íntegramente las distintas tecnologías que utilizamos para su desarrollo.
Por último, expondremos los resultados obtenidos de una serie de pruebas y mediciones que realizamos para evaluar el desempeño de nuestra solución ideada.

Este trabajo abarca un conjunto variado de tecnologías, algunas muy conocidas y estándar de la industria, otras de nicho y aplicación en problemas más particulares.
Desde motores de videojuegos, paradigmas de programación concurrente, principios de sistemas distribuidos, hasta sistemas de orquestación de contenedores, veremos cómo
relacionamos todas estas tecnologías para resolver el problema de estudio.

Como estudiantes de la carrera de Ingeniería en Informática todos los integrantes de este grupo tenemos experiencia, y particular aprecio, por los videojuegos.
Es un pasatiempo que nos ha acompañado además a lo largo de nuestras vidas y que nos resulta de amplio interés. Es por esto mismo que el objeto de este trabajo
se centra, en particular, en el desarrollo de videojuegos en línea. 
