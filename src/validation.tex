% Aquí se debe definir un conjunto de pruebas que serán entregadas como 
% resultado final del proyecto y verificación del mismo. 
% Las mismas podrán ser ajustadas más adelante.

\noindent Inicialmente, partimos de la premisa de que nuestra arquitectura distribuida era naturalmente más escalable horizontalmente
que una arquitectura monolítica. En esta sección realizaremos diversas pruebas con distintas configuraciones de nodos del servidor y cantidad
de jugadores conectados para validar esta hipótesis. Para facilitar estas pruebas y hacerlas reproducibles, utilizaremos los Bots desarrollados
en la sección \ref{sec:Bots}. Las pruebas fueron realizadas en el entorno de Kubernetes descrito en la sección TODO, que es actualmente el entorno
productivo de \textit{Fiubakka}, pero es posible realizar estas pruebas sin necesidad de un ambiente de Kubernetes, mediante múltiples instancias del
servidor en una misma máquina.

Como aclaración previa a las pruebas, el cluster de Kubernetes utilizado cuenta con un nodo con una CPU aproximadamente el doble de rápida que los otros dos nodos.
Todas las pruebas que involucren menos de la totalidad de los nodos únicamente utilizarán los nodos más lentos para mantener uniformidad en los resultados.
Igualmente, a los efectos del resultado esperado, es esperable que dada una utilización de CPU \textit{x} en un nodo del servidor, se reduzca aproximadamente a
$\frac{x}{n}$ donde \textit{n} es la cantidad de nodos del Akka cluster. Esta proporción es independiente de los recursos de cada nodo de Kubernetes.

\subsection{Recursos utilizados por un nodo de \textit{Fiubakka}}

\noindent Comenzamos con la prueba más simple, un único nodo del cluster de \textit{Fiubakka} en \textit{idle}, esto es, sin ningún jugador conectado.



\subsection{Resultados esperados}

\subsection{Resultados obtenidos}

\subsection{Métricas de latencia}
