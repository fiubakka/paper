% En esta sección se debe definir todo el proceso de construcción de software,
% elegir metodología de trabajo, entregables, hitos de avance, 
% pruebas de código, etc. Se debe explicar la metodología elegida y 
% su adaptación al trabajo particular. En la descripción de la metodología 
% se debe incluir una descripción de cómo se gestiona el alcance, tiempos, 
% estimaciones, indicadores, riesgos, calidad, reuniones dentro del equipo y 
% con los interesados. Se hace especial hincapié en el hecho de la 
% documentación del sistema en cuestión, incluyendo documentación técnica de 
% entregables, documentación funcional y de diseño, minutas de reuniones, etc.

\noindent A continuación se detalla el cronograma del avance del proyecto,
definiendo cada hito y las fechas aproximadas de inicio y fin de cada uno.
Estos hitos fueron acompañados de reuniones semanales entre los integrantes,
además de reuniones también semanales o bisemanales con el tutor, para mostrar y
validar el avance.

\begin{longtable}{|>{\centering\arraybackslash}p{3cm}|>{\centering\arraybackslash}p{\dimexpr\textwidth-4cm\relax}|}
    \hline
    \textbf{Fecha} & \textbf{Tareas} \\ \hline
    \endfirsthead
    \hline
    \textbf{Fecha} & \textbf{Tareas} \\ \hline
    \endhead
    \hline
    \endfoot
    \hline
    \endlastfoot

    \multirow{3}{3cm}{\centering 31/08/23 \\ -- \\ 16/09/23} &
    \begin{itemize}[left=0pt]
        \item Inicio de la cursada del Trabajo Profesional.
        \item Asignación de tutor.
        \item Reuniones para discutir qué tipo de proyecto se desea crear.
    \end{itemize} \\ \hline

    \multirow{2}{3cm}{\centering 17/09/23 \\ -- \\ 18/10/23} &
    \begin{itemize}[left=0pt]
        \item Creación de la propuesta del proyecto.
        \item Investigaciones iniciales.
        \item Primeras reuniones con el tutor.
    \end{itemize} \\ \hline

    \multirow{3}{3cm}{\centering 19/10/23 \\ -- \\ 9/11/23} &
    \begin{itemize}[left=0pt]
        \item Investigaciones generales sobre las tecnologías que vamos a utilizar en un inicio en nuestro proyecto.
    \end{itemize} \\ \hline

    \multirow{3}{3cm}{\centering 10/11/23 \\ -- \\ 6/12/23} &
    \begin{itemize}[left=0pt]
        \item Realizar la configuración inicial de las tecnologías necesarias para el desarrollo del trabajo, es decir, configurar en Godot un cliente y levantarlo en un entorno local y configurar un servidor con Akka y levantarlo también en el ambiente local.
        \item Asignación de roles iniciales para los integrantes del grupo.
        \item Aplicación mínima de comunicación entre el servidor y el cliente.
        \item Prueba de concepto del modelo propuesto en Akka. Aplicación que instancie actores y la lógica de event streaming descrita en la arquitectura propuesta.
        \item Definición del protocolo de comunicación entre el servidor y el cliente.
        \item Configuración de Akka Streams y Akka Cluster.
        \item Configuración de la base de datos.
        \item Configuración de Dagger.
        \item Inicio en el deploy de Kubernetes.
        \item Configuración de Kafka.
        \item Experimentación de la comunicación de Godot con un servidor de prueba en python.

    \end{itemize} \\ \hline

    \multirow{3}{3cm}{\centering 7/12/23 \\ -- \\ 31/12/23} &
    \begin{itemize}[left=0pt]
        \item Implementación en Akka de las acciones básicas de los jugadores y las interacciones entre ellos.
        \item Implementación inicial en Godot de las acciones básicas entre jugadores. Ejemplo: que los usuarios puedan mover su propio personaje y ver las acciones de otros jugadores y que puedan comunicarse a través de un chat.
        \item Implementación inicial de un mapa con el cual los jugadores puedan interaccionar. Ejemplo: colisiones y la visualización del entorno donde se encuentra el jugador.
        \item Creación de pantallas para ingresar al juego y para registrarse.
        \item Configuración de Cinnamon.
        \item Configuración de Telemetry.
        \item Configuración de la base de datos en producción.
        \item Creación de jugadores artificiales que simulen acciones sencillas para testeo.
        \item Mejora en el manejo de la desconexión de jugadores.

    \end{itemize} \\ \hline

    \multirow{3}{3cm}{\centering 1/01/24 \\ -- \\ 25/01/24} &
    \begin{itemize}[left=0pt]
        \item Implementación de un inventario para equipar cosméticos al jugador.
        \item Integración de las distintas pantallas del juego.
        \item Implementación de un creador de personajes.
        \item Cambio de TCP a aeronUDP en Akka remoting.
    \end{itemize} \\ \hline

    \multirow{3}{3cm}{\centering 26/02/24 \\ -- \\ 05/03/24} &
    \begin{itemize}[left=0pt]
        \item Creación de una pantalla de inicio.
        \item Animaciones para los jugadores.
        \item Configuración de CloudKarafka.
    \end{itemize} \\ \hline

    \multirow{3}{3cm}{\centering 06/03/24 \\ -- \\ 14/04/24} &
    \begin{itemize}[left=0pt]
        \item Implementación de un controlador de escenas.
        \item Implementación del Truco.
        \item Mejoras en el aspecto multijugador (ej: manejar un jugador que se desconecta, que entra a otra habitación, etc.)
        \item Configuración de CircleCI.
        \item Migración a de Scala 2.13 a Scala 3.4.
        \item Mejora en el deploy.
    \end{itemize} \\ \hline

    \multirow{3}{3cm}{\centering 15/04/24 \\ -- \\ 09/05/24} &
    \begin{itemize}[left=0pt]
        \item Mejoras en las pantallas de inicio de sesión y registro.
        \item Implementación de una pantalla de carga.
        \item Creación de personajes no jugables y una caja de diálogos para que los jugadores puedan comunicarse con ellos.
        \item Implementación de dos idiomas para elegir.
        \item Continuación del Truco.
        \item Informe.        
    \end{itemize} \\ \hline

    \multirow{3}{3cm}{\centering 10/05/24 \\ -- \\ 10/06/24} &
    \begin{itemize}[left=0pt]
        \item Pausar la música.
        \item Mejorar los límites de la cámara para los bordes de los mapas.
        \item Más personajes no jugables para poblar los mapas.
        \item Controlador de música.
        \item Continuación del Truco.
        \item Informe.
    \end{itemize} \\ \hline

    \multirow{3}{3cm}{\centering 10/06/24 \\ -- \\ 30/06/24} &
    \begin{itemize}[left=0pt]
        \item Optimización del uso de los sprite sheets para los personajes.
        \item Detalles en los mapas.
        \item Sonidos en los menús.
        \item Informe.
    \end{itemize} \\ \hline
\end{longtable}
