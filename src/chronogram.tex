% En esta sección se debe definir todo el proceso de construcción de software,
% elegir metodología de trabajo, entregables, hitos de avance, 
% pruebas de código, etc. Se debe explicar la metodología elegida y 
% su adaptación al trabajo particular. En la descripción de la metodología 
% se debe incluir una descripción de cómo se gestiona el alcance, tiempos, 
% estimaciones, indicadores, riesgos, calidad, reuniones dentro del equipo y 
% con los interesados. Se hace especial hincapié en el hecho de la 
% documentación del sistema en cuestión, incluyendo documentación técnica de 
% entregables, documentación funcional y de diseño, minutas de reuniones, etc.

\subsection{Cronograma de actividades}

\begin{center}
\begin{tabularx}{\textwidth}{
    | >{\centering\arraybackslash}X 
    | >{\centering\arraybackslash}X | }
        \hline
        \textbf{Fecha} & \textbf{Tareas} \\
        \hline
        \multirow{4}{*}{31/08/23 - 16/09/23} &
        \begin{itemize}
            \item Inicio de la cursada del Trabajo Profesional
            \item Asignación de tutor
            \item Reuniones para discutir qué tipo de proyecto se desea crear
        \end{itemize} \\
        \hline

        \multirow{4}{*}{17/09/23 - 18/10/23} &
        \begin{itemize}
            \item Creación de la propuesta del proyecto
            \item Investigaciones iniciales
            \item Primeras reuniones con el tutor
        \end{itemize} \\
        \hline

        \multirow{4}{*}{19/10/23 - 9/11/23} &
        \begin{itemize}
            \item Investigaciones generales sobre las tecnologías que vamos a utilizar en un inicio en nuestro proyecto
        \end{itemize} \\
        \hline

        \multirow{4}{*}{10/11/23 - 6/12/23} &
        \begin{itemize}
            \item Realizar la configuración inicial de las tecnologías necesarias para el desarrollo del trabajo, es decir, configurar en Godot un cliente y levantarlo en un entorno local y configurar un servidor con Akka y levantarlo también en el ambiente local.
            \item Asignación de roles iniciales para los integrantes del grupo.
            \item Aplicación mínima de comunicación entre el cliente y el servidor.
            \item Prueba de concepto del modelo propuesto en Akka. Aplicación que cree actores y la lógica de event streaming descrita en la arquitectura propuesta.
            \item Definición del protocolo de comunicación entre el servidor y el cliente.
        \end{itemize} \\
        \hline

        \multirow{4}{*}{7/12/23 - 31/12/23} &
        \begin{itemize}
            \item Implementación en Akka de las acciones básicas de los jugadores y las interacciones entre ellos.
            \item Implementación inicial en Godot de las acciones básicas entre jugadores. Ejemplo: que los usuarios puedan mover su propio personaje y ver las acciones de otros jugadores y que puedan comunicarse a través de un chat.
            \item Implementación inicial de un mapa con el cual los jugadores puedan interaccionar. Ejemplo: colisiones y la visualización del entorno donde se encuentra el jugador.
            \item Creación de pantallas para ingresar al juego y para registrarse.
        \end{itemize} \\
        \hline

    % \multirow{3}{3cm}{\centering 1/01/24 \\ -- \\ 25/01/24} &
    % \begin{itemize}
    %     \item Implementación de un inventario para equipar cosméticos al jugador.
    %     \item Integración de las distintas pantallas del juego.
    %     \item Implementación de un creador de personajes.        
    % \end{itemize} \\ \cline{2-2}
    % \hline

    % \multirow{3}{3cm}{\centering 26/02/24 \\ -- \\ 05/03/24} &
    % \begin{itemize}
    %     \item Creación de una pantalla de inicio.
    %     \item Animaciones para los jugadores.
    % \end{itemize} \\ \cline{2-2}
    % \hline

    % \multirow{3}{3cm}{\centering 06/03/24 \\ -- \\ 14/04/24} &
    % \begin{itemize}
    %     \item Implementación de un controlador de escenas.
    %     \item Implementación del Truco.
    %     \item Mejoras en el aspecto multijugador (ej: manejar un jugador que se desconecta, que entra a otra habitación, etc.)
    % \end{itemize} \\ \cline{2-2}
    % \hline

    % \multirow{3}{3cm}{\centering 15/04/24 \\ -- \\ 09/05/24} &
    % \begin{itemize}
    %     \item Mejoras en las pantallas de inicio de sesión y registro.
    %     \item Implementación de una pantalla de carga.
    %     \item Creación de personajes no jugables y una caja de diálogos para que los jugadores puedan comunicarse con ellos.
    %     \item Implementación de dos idiomas para elegir.
    %     \item Continuación del Truco.
    %     \item Informe.        
    % \end{itemize} \\ \cline{2-2}
    % \hline

    % \multirow{3}{3cm}{\centering 10/05/24 \\ -- \\ 10/06/24} &
    % \begin{itemize}
    %     \item Pausar la música.
    %     \item Mejorar los límites de la cámara para los bordes de los mapas.
    %     \item Más personajes no jugables para poblar los mapas.
    %     \item Controlador de música.
    %     \item Continuación del Truco.
    %     \item Informe.
    % \end{itemize} \\ \cline{2-2}
    % \hline

    % \multirow{3}{3cm}{\centering 10/06/24 \\ -- \\ 30/06/24} &
    % \begin{itemize}
    %     \item Optimización del uso de los sprite sheets para los personajes.
    %     \item Detalles en los mapas.
    %     \item Sonidos en los menús.
    %     \item Informe.
    % \end{itemize} \\ \cline{2-2}
    \hline
\end{tabularx}
\end{center}
