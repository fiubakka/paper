% Reference using APA 6 format: https://biblioguias.uma.es/citasybibliografia/ejemplosAPA
% Internet resources:
% Apellidos, N. (año) o (año, mes) o (año, día de mes). Título del artículo/sección. Nombre de web. URL
%Organización Mundial de la Salud. (2021, 17 de noviembre). Salud mental del adolescente. Organización Mundial de la Salud. https://www.who.int/es/news-room/fact-sheets/detail/adolescent-mental-health 

\begin{thebibliography}{99}
    \bibitem{ref1} Juan Linietsky, Ariel Manzur and the Godot community. (2024). \textit{Introduction to Godot}. Godot Docs. \url{https://docs.godotengine.org/en/stable/getting\_started/introduction/introduction\_to\_godot.html#how-does-it-work-and-look}
    \bibitem{ref3} Juan Linietsky, Ariel Manzur and the Godot community. (2024). \textit{System requirements}. Godot Docs. \url{https://docs.godotengine.org/en/stable/about/system_requirements.html#id1}
    \bibitem{ref2} Juan Linietsky, Ariel Manzur and the Godot community. (2024). \textit{Importing images}. Godot Docs. \url{https://docs.godotengine.org/en/stable/tutorials/assets\_pipeline/importing\_images.html#compress-mode}
    \bibitem{moores-law-is-dead} Audrey Woods. (2021). \textit{The Death of Moore\'s Law: What it means and what might fill the gap going forward}. MIT CSAIL Alliances. \url{https://cap.csail.mit.edu/death-moores-law-what-it-means-and-what-might-fill-gap-going-forward}
    
\end{thebibliography}
