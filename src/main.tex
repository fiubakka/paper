\documentclass[12pt]{article}

\usepackage[spanish]{babel} % Remove this line if you want English language support
\usepackage[hyperindex]{hyperref}
\usepackage{graphicx}
\usepackage{enumitem}
\usepackage{subfiles} % Best loaded last in the preamble

\addto\captionsspanish{\renewcommand{\contentsname}{Tabla de Contenidos}}

\begin{document}

\title{\textbf{FIUBAKKA} \\ \large \textbf{Juego multijugador online con Akka}}
\author{
    Rolando, Marcos
    \and
    Soriano, Iván
    \and
    Raveszani, Nicole
    \and
    Scaccheri Cassanello, Franco
    \and
    Deymonnaz, Pablo
}
\date{\today}

\maketitle % Prints the title, author, and date

\thispagestyle{empty}

\begin{figure}[htbp]
    \centering
    \includegraphics[width=0.5\textwidth]{../assets/fiuba-logo.png}
\end{figure}
\begin{figure}[htbp]
    \centering
    \includegraphics[width=0.5\textwidth]{../assets/akka-toolkit-logo.png}
\end{figure}

\newpage
\thispagestyle{empty}
\tableofcontents
\newpage

\setcounter{page}{1} % Reset page counter to 1

\section{Resumen / Abstract}

\subfile{resumen.tex}

\vspace{5mm} %5mm vertical space

\subfile{abstract.tex}

\newpage

\section{Introducción}

\subfile{introduction.tex}

\section{Backend: servidor distribuido Akka}
\index{Desarrollo}

\subsection{Introducción a Akka}

\subsection{Arquitectura}

\subsection{Diagramas}

\subsection{Mensajes de Kafka}


\section{Frontend: cliente Godot}

\subsection{Introducción a Godot}

\subsection{Estructura de proyecto}

\subsection{Comunicación con el servidor}

\subsection{Dificultades}


\section{Resultados}

\subsection{Resultados esperados}

\subsection{Resultados obtenidos}

\subsection{Métricas de latencia}

\section{Conclusiones}



\end{document}
