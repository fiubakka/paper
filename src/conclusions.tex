% Las conclusiones deben recapitular los aspectos principales descritos en 
% las secciones anteriores. Es importante que quede claro en qué contribuyó 
% el Trabajo Profesional a su formación, a otras entidades si las hubiera, 
% y los resultados principales obtenidos. También se debe destacar si el 
% desarrollo generó alguna innovación y de qué tipo.

Dados los resultados presentados en la sección \ref{sec:validation}, podemos concluir que la premisa
en la cual se basó nuestro proyecto, que propone que una arquitectura distribuida es beneficiosa para un
videojuego con una gran cantidad de jugadores dadas las facilidades que presenta a la hora de escalar,
podemos concluir que nuestro proyecto fue exitoso. Logramos demostrar que al escalar horizontalmente,
es decir, al aumentar la cantidad de nodos del servidor en lugar de aumentar la potencia de los nodos existentes,
la carga disminuye, lo cual nos permitiría tener una cantidad mucho mayor de jugadores que si solo tuvieramos
una cantidad fija de nodos y solo pudieramos mejorar su hardware.
Sin embargo, hay múltiples consideraciones a tener en cuenta dada nuestra arquitectura. Principalmente, los nodos
del servidor deben tener un hardware lo suficientemente potente como para poder aprovechar las ventajas de la
distribución, como fue explicado en mas detalle en la sección \ref{sec:virtualization}, lo cuál puede incurrir en 
costos considerablemente altos.

Otra cosa a tener en cuenta es el tipo de juego sobre el cual se quiere aplicar esta arquitectura. Dada la naturaleza
distribuida, como ya fue explicado en secciones anteriores, la consistencia de la información para todos los jugadores
será eventual, es decir, en un determinado instante dos jugadores pueden tener información distinta, pero dado cierto tiempo,
todos terminarán sincronizados. Esto conlleva a que los juegos de naturaleza más competitiva no se adaptarían bien a esta
arquitectura, debido a estos problemas de sincronización, o también posibles modificaciones en el cliente que le den una ventaja
injusta a los jugadores que lo hagan.
Otro tipo de juegos incompatibles con nuestra arquitectura son aquellos que requieran una latencia extremadamente baja, como sería
un juego de disparos en primera persona. Existen algunas partes en la arquitectura que agregan demasiado overhead y harían que la
experiencia en este tipo de juegos no sea optima.

La experiencia de llevar a cabo este proyecto fue extremadamente enriquecedora para todos los integrantes del grupo.
Nos permitió poner en práctica, una gran cantidad de conceptos aprendidos durante el transcurso de la carrera como algoritmos convencionales,
, programación concurrente, arquitecturas distribuidas, comunicación sobre la red, complejidad algorítmica, escalabilidad e incluso metodologías
de desarrollo ágiles y administración de proyectos. Si bien nos dividimos en equipos durante el desarrollo, eventualmente todos aportamos
en cada una de las áreas, en distinta medida, pero lo suficiente para que tengamos un entendimiento de todos los conceptos aplicados.

Para concluir, consideramos que nuestro proyecto culminó en una arquitectura que es una base muy sólida para el desarrollo de videojuegos
que busquen manejar una gran cantidad de jugadores interactuando entre sí. Creemos que esto puede abrir la puerta al desarrollo de más juegos
de este estilo.